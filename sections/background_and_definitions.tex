\section{Background and definitions}
\label{subsec:background}

As envisaged by Sir Tim Berners-Lee, the Web is moving from an interlinked documents space to a global information one where both documents and data are linked: The Semantic Web \cite{berners2001semantic}. Related to this Semantic Web, Linked Data is a set of best practices to publish data on the Web in a machine-readable way, with a explicitly defined semantic meaning, linked to other datasets and allowed to be searched for \cite{bizer2009linked}. In 2006, Sir Tim Berners-Lee described a set of principles to publish Linked Data on the Web:

\begin{enumerate}
  \item Use URIs as names for things
  \item Use HTTP URIs so that people can look up those names
  \item When someone looks up a URI, provide useful information, using the standard (RDF, SPARQL)
  \item Include links to other URIs, so that they can discover more things \\[\baselineskip]
\end{enumerate}

Later, in 2010, he established a five-star rating system to encourage people and governments to publish high-quality Linked Open Data:\\[\baselineskip]

\begin{tabular}{ l p{0.75\linewidth} }
  $\star$ & Available on the web (whatever format) but with an \textbf{open licence}, to be Open Data. \\
  $\star$ $\star$ & Available as machine-readable structured data (e.g. excel instead of image scan of a table). \\
  $\star$ $\star$ $\star$ & as (2) plus non-propietary format (e.g. CSV instead of excel). \\
  $\star$ $\star$ $\star$ $\star$ & All the above plus: Use open standards from W3C (RDF and SPARQL) to identify things, so that people can point at your stuff. \\
  $\star$ $\star$ $\star$ $\star$ $\star$ & All the above plus: Link your data to other people's data to provide context. \\[\baselineskip]
\end{tabular}

Over the years, enormous amount of applications based on Linked Data have been developed. Big companies like Google\footnote{\url{http://www.google.com/insidesearch/features/search/knowledge.html}}, Yahoo!\footnote{\url{http://semsearch.yahoo.com/}} or Facebook\footnote{\url{http://ogp.me/}} have lately invested resources on deploying Semantic Web and Linked Data technologies. Several governments from countries like the United States of America or the United Kingdom have published a big amount of Open Data from different administrations in their Open Data portals.

Linked Data is real and can be the key for data management in smart cities. Following these recommendations, data publishers can move towards a new data-powered space, in which data scientists and application developers can research on new uses for Linked Open Data.
