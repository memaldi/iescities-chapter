\section{Identified challenges}

As previously mentioned, data management in smart cities is becoming a difficult task due to:
\begin{itemize}
	\item Volume
	\item Variety
	\item Velocity
\end{itemize}

This three variables can also be found in \textit{Big Data}-related articles, so it's not surprising at all that smart cities are going to deal with Big Data problems in the near future (if they are not dealing with them right now).

\subsection{Volume}

The high amount of data used and generated by cities nowadays needs to be properly analysed, processed, stored and eventually accessible. This means conventional IT structures need to evolve, enabling scalable storage technologies, distributed querying approaches and massively parallel processing algorithms and architectures.

% Apache Hadoop
% MapReduce
% NonSQL

However, big amounts of data should not be seen as a drawback attached to smart cities. The larger the datasets, the better analysis algorithms can perform, so deeper insights and conclusions should be expected as an outcome. These could ease the decission making stage.

\subsection{Variety}

Data is rarely found in a perfectly ordered and ready for processing format. Data scientist are used to work with diverse sources, which seldom fall into neat relational structures.

% talk a little bit about data quality, provenance, etc.

\subsection{Velocity}

Finally, we must assume that data generation is experiencing an exponential growth. That forces our IT structure to not only tackle with volume issues, but with processing rates. A widely spread concept among data businesses is that sometimes you can not rely on five-minute-old data for your business logic.

That's why \textit{streaming data} has moved from academic fields to industry to solve velocity problems. There are two main reasons to consider streaming processing:
\begin{itemize}
	\item Sometimes, input data is too fast to store in their entirety without rocketing costs.
	% At the extreme end of the scale, the Large Hadron Collider at CERN generates so much data that scientists must discard the overwhelming majority of it — hoping hard they’ve not thrown away anything usefu
	\item If applications mandate immediate response to the data, batch processes are not suitable. Due to the rise of smartphone applications, this trend is increasingly becoming a common scenario.
\end{itemize}