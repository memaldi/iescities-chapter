\section{Introduction}
\label{sec:introduction}

In the last decade, cities have been sensorized, protocols are constantly refined to deal with the possibilities that new hardware offers, communication networks are offered in all flavours, etc., generating lots of data that needs to be dealt with. Public administrations are not usually able to process all the data they have in an efficient way, resulting in large amounts of data going un-analized and affecting the end-users that could benefit from them.

Citizens are also being encouraged to adopt the role of linked open data providers. User-friendly Linked Data apps should allow citizens to easily contribute with new trustable data that can be linked to already existing published (more static generally) Linked Open Data provided by city councils. In addition, people-centric mobile sensing, empowered by the technology inside actual smartphones, should progress into continous people-centric enriched Linked Data. Linked Open Data also encourages the linkage to other resources described formally through structured vocabularies, allowing the discovery of related information and the possibility to make inferences, resulting in higher quality data.

Data management is becoming one of the greatest challenges of the $21^{st}$ century. Regarding urban growth, experts predict that global urban population will double by the year 2050, meaning that nearly 70\% of the whole planet's inhabitants will be living in a major town or city.

This prediction arises the need to deal with the huge amounts of data generated by cities, enabling the possibility to manage their resources in an efficient way. The \textit{Smart Data} term has been coined to address the data that makes itself understandable, by extracting relevant information and insights from big data and presenting the conclusions as human-friendly visualizations.

The problems of managing data are moving to a new level. It's not only a matter of caring about \textit{more} data, but how we can use it efficiently in our processes. It's about how we can deal with increasing volumes of data (from standalone databases to real \textit{Big Data}) and integrate them to our advantage, making it useful and digestible in order to make better decisions.

In the last few years, the \textit{Smart city} concept has been adopted to refer to those cities aware of their citizens' life quality, worried about the efficiency and trustworthiness of the services provided by governing entities and businesses.

Smart data (understood as curated, high-quality and digestible data) can help cities reach a \textit{Smart City} status, analysing the generated data streams and providing useful information to their users: citizens, council managers, third parties, etc.

Although, efficient data lifecycle management processes need to be adopted as best practices, avoiding to convert input data in non-sense noise that can not be used to improve council's services.

Thus, our approach is based on an actual review of the state of the art regarding data lifecycle management, proposing our own model as a more refined approach to the existing ones. We also encourage the adoption of Linked Open Data principles to publish both the whole generated data and the processed data, in order to allow further research on the area by third parties and the development of new business models relying on public access data. Similar proposals regarding Linked Data are defended by \cite{Bizer:2012:MUB:2094114.2094129}.
