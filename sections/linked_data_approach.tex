\section{Open Linked data as a viable approach}

In the previous section, we identified some of the challenges smart cities will need to face in the following years. The data lifecycle model proposed at Figure \ref{fig:model} relies on Linked Open Data principles to try to solve these issues, reducing costs and enabling third parties to develop new business models on top of Linked Open Data.

Next we describe how Linked Open Data principles could help in the model's stages:

\subsection{Capture}

\subsection{Process}

\subsection{Store}

\subsection{Publish}

\subsection{Discovery}

\subsection{Consume}

% Once the data is published, we use the provided methods to consume the data. This data consumption involves the data mining, analytics or reasoning.

At this stage, we focus on consuming data for our logic processes, should they involve data mining algorithms, analytics, reasoning, etc.

Whereas complex processing algorithms can be used independently of the dataset format, Linked Open Data can greatly help at reasoning purposes. Linked Open Data allows to describe entities using constraints and restriction rules (belonging, domain, range, etc.), favoring the inference of new information from the existing one. Thanks to Linked Data, we are not feeding our algorithms with raw data (numbers, strings, values...), but with semantically meaningful data (height in cm, world countries, company names...).

Using Linked Data we can make use of semantics to enrich input data and the processing algorithms, resulting in higher quality outputs.

While dealing with Linked Data streams, stream-querying languages such as CQELS's\footnote{\url{https://code.google.com/p/cqels/}} language (an extension of the declarative SPARQL 1.1 language using the EBNF notation) can greatly help in the task. CQELS\cite{le2011native} (Continuous Query Evaluation over Linked Stream) is a native and adaptive query processor for unified query processing over Linked Stream Data and Linked Data developed at DERI Galway\footnote{http://www.deri.ie/}.

\subsection{Visualize}

In order to make meaning from data, humans have developed a great ability to understand visual representations. The main objective of data visualization is to communicate information in a clean and effective way through graphical means. It's also suggested that visualization should also encourage users engagement and attention.

The \textit{"A picture is worth a thousand words"} saying reflects the power images and graphics have when expressing information.

As Linked Data is based on subject-predicate-object triples, graphs are a natural way to represent triple stores, where subject and object nodes are inter-connected through predicate links. When further analysis is applied on triples, a diverse variety of representations can be chosen to show processed information: charts, infographics, flows, etc.