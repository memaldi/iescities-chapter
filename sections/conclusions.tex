\section{Conclusions}

In this chapter we have proposed Linked Data as a suitable paradigm to manage the entire data life cycle in Smart Cities. As can be seen, along this chapter we expose a set of guidelines for public or private managers which want to contribute with data from their administration or enterprise into a Smart City, bringing closer existing tools and exposing practical knowledge acquired by authors while working with Linked Data technologies. The proposed data life cycle for Smart Cities covers the entire travelling path of data inside a Smart City, and the mentioned tools and technologies fulfil all the needed tasks to go forward on this path.

But Linked Open Data is not all about technology. The \textit{Open} term of Linked Open Data is about the awareness of public (and private) administrations to provide citizens with all the data which belong to them, making the governance process more transparent; the awareness of developers to discover the gold behind data and the awareness of fully informed citizens participating on decision making processes: Smart Cities, smart business and Smart Citizens.

Urban Linked Data applications also empower citizens' role of first level data providers. Thanks to smartphones, each citizen is equipped with a full set of sensors which are able to measure the city's pulse at every moment: traffic status, speed of each vehicle to identify how they are moving, reporting of roadworks or malfunctioning public systems, and so forth. Citizens are moving from data consumers to data \textit{prosumers}, an aspect data scientists and application developers can benefit from to provide new services for Smart Cities.
