\section{Conclusions}

In this chapter we have proposed Linked Data as a suitable paradigm to manage the entire life cycle of data in smart cities. As can be seen, along this chapter we expose a set of guidelines for public or private managers, which want to introduce data from their administration or enterprise into an smart city, bringing closer existing tools and exposing practical knowledge acquired by authors while working with Linked Data technologies. Proposed data life cycle for smart cities, covers entire travelling path of data inside a smart city, and proposed tools fulfils all tasks needed to go forward on this path.

But Linked Open Data is not all about technology. The \textit{Open} term of Linked Open Data is about awareness of public (and private) administrations to provide citizens with all the data which belong to them, making the governance process more transparent; awareness of developers to discover the gold behind data and awareness of fully informed citizens participating on decision making: smart cities, smart business and smart citizens.
